The K-NN algorithm stores the features of the train data-set and computes the distance of the examples to be classified from the K nearest neighbors. The pros include that it is really easy to implement, and that the data-set can be provided incrementally (the \textit{lazy learning} approach analyzes the data only when it is used for classification). The downsides of the algorithm are that the classification is slow (due to the lazy approach), and that it is sensible to feature selection.